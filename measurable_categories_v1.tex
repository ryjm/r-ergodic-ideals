\documentclass[letterpaper,10pt,oneside,onecolumn,reqno]{amsart}
%leqno = left hand equation numbering
%oneside vs. twoside = how many pages are you looking at at a time?
%onecolumn vs. twocolumn = obvious

%%% Packages
\usepackage{amsmath, amssymb}
\usepackage[left=1in,top=1in,right=1in,bottom=1in]{geometry}
\usepackage{setspace}
\usepackage{hyperref}
\usepackage{enumerate}
\usepackage{wrapfig}
\usepackage{graphicx}
\usepackage[all]{xy}
\usepackage{color}
\usepackage[usenames,dvipsnames]{xcolor}
\usepackage{refcheck}

\onehalfspace
\setcounter{tocdepth}{2}	% If this equals 1, the table of contents does not include subsections
%\includeonly{}

%\def\singlespaced{\baselineskip=\normalbaselineskip}


%%% Sets
\newcommand{\A}{\mathcal A}
\newcommand{\B}{\mathcal B}
\newcommand{\C}{\mathcal C}
\newcommand{\D}{D}
\newcommand{\E}{\mathbb E}
\newcommand{\F}{\mathcal F}
\renewcommand{\H}{\mathcal H}
\newcommand{\I}{\mathcal I}
\newcommand{\K}{\mathcal K}
\newcommand{\M}{\mathcal M}
%\newcommand{\MM}{\operatorname{M}}
\newcommand{\N}{\mathcal N}
\newcommand{\NN}{\mathbb N}
\renewcommand{\P}{\mathbb P}
\newcommand{\PP}{\mathbb P}
\newcommand{\Q}{\mathbf Q}
\newcommand{\R}{\mathbb R}
\newcommand{\T}{\mathcal T}
\newcommand{\V}{\mathcal V}
\newcommand{\W}{\mathcal W}
\newcommand{\X}{\mathcal X}
\newcommand{\Z}{\mathbb Z}

\renewcommand{\c}{{\operatorname{c}}}
\newcommand{\e}{\mathrm{e}}				% Euler's constant.
\renewcommand{\d}{\mathrm{d}}				% Differential operator sans space.
\newcommand{\sd}{\, \mathrm{d}}		% Differential operator with space.
\renewcommand{\i}{\mathrm{i}}				% Square root of $-1$.
\newcommand{\oo}{\infty}
\newcommand{\Var}{\operatorname{Var}}

\newcommand{\tr}{\operatorname{tr}}

\newcommand{\ent}{\operatorname{ent}}

\newcommand{\arginf}{\operatorname{arginf}}
\newcommand{\argsup}{\operatorname{argsup}}

% Environments
\theoremstyle{definition}
\newtheorem{thm}{Theorem}
\newtheorem{cor}[thm]{Corollary}
\newtheorem{lem}[thm]{Lemma}
\newtheorem{pro}[thm]{Proposition}
\newtheorem{conj}[thm]{Conjecture}
\newtheorem{defn}[thm]{Definition}
\newtheorem{ass}[thm]{Assumption}
\newtheorem{rem}[thm]{Remark}
\newtheorem{exa}[thm]{Example}
%\theoremstyle{alpha}
\newtheorem{hyp}{Hypothesis}


\newcommand{\End}{\operatorname{End}}
\newcommand{\Hom}{\operatorname{Hom}}
\newcommand{\Aut}{\operatorname{Aut}}
\newcommand{\InnAut}{\operatorname{InnAut}}

\newcommand{\Euc}{\operatorname{Euc}}
\renewcommand{\O}{\operatorname{O}}
\newcommand{\SO}{\operatorname{SO}}

\newcommand{\catname}[1]{{\normalfont\textbf{#1}}}
\newcommand{\DynSys}{\catname{DynSys}}
\newcommand{\Set}{\catname{Set}}
\newcommand{\Meas}{\catname{Meas}}
\newcommand{\Top}{\catname{Top}}
\newcommand{\LieGrp}{\catname{LieGrp}}
\newcommand{\HomSp}{\catname{HomSp}}
\newcommand{\HomMan}{\catname{HomMan}}
%\newcommand{\SmoothMan}{\catname{SmoothMan}}
\newcommand{\Man}{\catname{Man}}
\newcommand{\VectBund}{\catname{VectBund}}
\newcommand{\Klein}{\catname{Klein}}

\newcommand{\notzero}{\emptyset}
%\newcommand{\notzero}{{\!\not{\,0}}}

\renewcommand{\bar}[1]{\overline{#1}}
\renewcommand{\hat}[1]{\widehat{#1}}
\renewcommand{\tilde}[1]{\widetilde{#1}}

\newcommand{\meet}{\wedge}
\newcommand{\join}{\vee}
\newcommand{\eval}{\operatorname{eval}}
\newcommand{\cov}{\operatorname{cov}}
\newcommand{\var}{\operatorname{var}}
\newcommand{\std}{\operatorname{std}}
\newcommand{\esssup}{\operatorname{ess\,sup}}

\newcommand{\lleq}{\ \underline{\ll}\ }

\renewcommand{\a}{\mathbf{a}}
\newcommand{\va}{\mathbf{\check a}}
\newcommand{\q}{\mathbf{q}}
\newcommand{\vvarphi}{\check \varphi}

\newcommand{\daggerdagger}{{\dagger\dagger}}

\newcommand{\symdiff}{{\,\triangle\,}}

%\newcommand{\tom}[1]{\textbf{{\color{ForestGreen}#1}}}

\newcommand{\tom}[1]{{\textbf{\color{ForestGreen}\#}{\color{ForestGreen}\{\emph{#1}\}}}}
\newcommand{\jake}[1]{{\textbf{\color{ForestGreen}\#}{\color{ForestGreen}\{\emph{#1}\}}}}

%\newcommand{\tom}[1]{\textbf{{\color{ForestGreen}\#}}\{#1\}}

\author{Tom LaGatta and Jake Miller}
\title{Ergodic Ideals}
\date{February 2013}



\begin{document}

\maketitle

	\textbf{Very rough draft. Please do not distribute.}

	Goal: ergodic averages from the structural viewpoint. 

	Measurable space $(X, \X)$, along with a measurable function $f : X \to X$. Assume that $\N$ is an ideal on $X$,\footnote{\tom{what is an ideal?}} and that the action of $f$ is ergodic with respect to $\N$.\footnote{\tom{what does this mean?}}
	
	
%	$\frac 1 N \sum_{n=1}^N \int_X \f( a^n(x) ) \, \P(\d x)$.
	
%	$\frac 1 N \sum_{n=1}^N f(a^n(x))$

	\section{Measure Formulation}

	Let $(X,\X)$ be a measurable space. \tom{Describe measurable subsets as ``events''.}
	
	Let $T = \NN$ be the natural numbers, and let $a^t : X \to X$ be a measurable action on the space.\footnote{That is, $a^{t'} \circ a^t = a^{t' + t}$ and $a^{0} = 1_X$, the identity map on $X$.} We define the inverse maps by $a^{-t} : \X \to \X$, noting that they go in the opposite direction. Formally, $a : T \to \End(X)$.
	
	
	
	Let $\P$ be a stationary, ergodic measure on $X$. As a measure, $\P : \X \to [0,1]$ is a countably additive function.
	
	Stationarity means that $a^t_* \P = \P$.\footnote{That is, $\P \circ a^{-t} = \P$.} Ergodic means that
		$$\mbox{if $\P( A \,\triangle\, a^{-t} A ) = 0$ for all $t \in T$, then $\P(A) = 0$ or $\P(A^c) = 0$.}$$

	\tom{What is a conditional expectation? $\E(f|\C)$ is a function $X \to \R$ which satisfies certain properties.} 
	
	
	

	\tom{Get to Birkhoff-Khinchin ergodic theorem}
	
		\begin{thm}[Birkhoff-Khinchin Ergodic Theorem]
			
		\end{thm}
		\begin{proof}
			
		\end{proof}
		
		\begin{cor}[Strong Law of Large Numbers]
		
		\end{cor}

	\section{Ideal Formulation}

	A subset $\N \subseteq \X$ is called an ideal if it is closed under countable unions, and is downward closed. That is, if $\{ N_i \} \subseteq \N$ is a countable set of measurable events, then $\bigcup N_i \in \N$; and if $N \in \N$ and $A \in \X$ with $A \subseteq N$, then $A \in \N$.

	We say that $\N$ is an ergodic ideal with respect to the $T$-action described by $a^t$ if:
		$$\mbox{if $A \,\triangle\, a^{-t} A \in \N$ for all $t \in T$, then $A \in \N$ or $A^c \in \N$.}$$


%	\section{Measurable Categories}

	


\end{document}		
		
		
		
		
		
		
		
		
		
		
		
		
