\documentclass[letterpaper,10pt,oneside,onecolumn,reqno]{amsart}
%leqno = left hand equation numbering
%oneside vs. twoside = how many pages are you looking at at a time?
%onecolumn vs. twocolumn = obvious

%%% Packages

\usepackage{amsmath, amssymb}
\makeindex
\usepackage[left=1in,top=1in,right=1in,bottom=1in]{geometry}
\usepackage{setspace}
\usepackage[colorlinks]{hyperref}
\usepackage{braket}
\usepackage{enumerate}
\usepackage{wrapfig}
\usepackage{graphicx}
\usepackage{framed}
\usepackage[all]{xy}
\usepackage{color}
\usepackage[usenames,dvipsnames]{xcolor}
%\usepackage{refcheck}
\usepackage{dsfont}             % for indicator function \mathds{one}

\onehalfspace
\setcounter{tocdepth}{2}	% If this equals 1, the table of contents does not include subsections
%\includeonly{}

%\def\singlespaced{\baselineskip=\normalbaselineskip}

% *** DEFLIST ****
%
% Aufruf: \begin{deflist}{Label}
%         \item[label] TeX
%         \end{deflist}
% erzeugt: Liste, bei der das laengste Label die Einrueckungstiefe bestimmt
%
%\def\deflabel#1{\bf #1\hfill}
\def\deflabel#1{#1\hfill}
\def\deflist#1{
%    \list{}{\settowidth{\labelwidth}{\bf #1}
    \list{}{\settowidth{\labelwidth}{#1}
            \setlength{\leftmargin}{\labelwidth}
            \addtolength{\leftmargin}{\labelsep}
            \let\makelabel \deflabel}}
\let\enddeflist\endlist
% END DEFLISTf

%%% Sets
\newcommand{\A}{\mathcal A}
\newcommand{\B}{\mathcal B}
\newcommand{\C}{\mathcal C}
\newcommand{\D}{D}
\newcommand{\E}{\mathbb E}
\newcommand{\F}{\mathcal F}
\renewcommand{\H}{\mathcal H}
\newcommand{\I}{\mathcal I}
\newcommand{\K}{\mathcal K}
\newcommand{\M}{\mathcal M}
%\newcommand{\MM}{\operatorname{M}}
\newcommand{\N}{\mathcal N}
\newcommand{\NN}{\mathbb N}
\renewcommand{\P}{\mathbb P}
\newcommand{\PP}{\mathbb P}
\newcommand{\Q}{\mathbf Q}
\newcommand{\R}{\mathbb R}
\newcommand{\T}{\mathcal T}
\newcommand{\V}{\mathcal V}
\newcommand{\W}{\mathcal W}
\newcommand{\X}{\mathcal X}
\newcommand{\Z}{\mathbb Z}

\newcommand{\one}{\mathds{1}}      % indicator function
\renewcommand{\c}{{\operatorname{c}}}
\newcommand{\e}{\mathrm{e}}				% Euler's constant.
\renewcommand{\d}{\mathrm{d}}				% Differential operator sans space.
\newcommand{\sd}{\, \mathrm{d}}		% Differential operator with space.
\renewcommand{\i}{\mathrm{i}}				% Square root of $-1$.
\newcommand{\oo}{\infty}
\newcommand{\Var}{\operatorname{Var}}

\newcommand{\tr}{\operatorname{tr}}

\newcommand{\ent}{\operatorname{ent}}

\newcommand{\arginf}{\operatorname{arginf}}
\newcommand{\argsup}{\operatorname{argsup}}

% Environments
\theoremstyle{definition}
\newtheorem{thm}{Theorem}
\newtheorem{cor}[thm]{Corollary}
\newtheorem{lem}[thm]{Lemma}
\newtheorem{pro}[thm]{Proposition}
\newtheorem{conj}[thm]{Conjecture}
\newtheorem{defn}{Definition}
\newtheorem{ass}[thm]{Assumption}
\newtheorem{rem}{Remark}
\newtheorem{exa}{Example}
%\theoremstyle{alpha}
\newtheorem{hyp}{Hypothesis}


\newcommand{\End}{\operatorname{End}}
\newcommand{\Hom}{\operatorname{Hom}}
\newcommand{\Aut}{\operatorname{Aut}}
\newcommand{\InnAut}{\operatorname{InnAut}}

\newcommand{\Euc}{\operatorname{Euc}}
\renewcommand{\O}{\operatorname{O}}
\newcommand{\SO}{\operatorname{SO}}

\newcommand{\catname}[1]{{\normalfont\textbf{#1}}}
\newcommand{\DynSys}{\catname{DynSys}}
%\newcommand{\Set}{\catname{Set}}
\newcommand{\Meas}{\catname{Meas}}
\newcommand{\Top}{\catname{Top}}
\newcommand{\LieGrp}{\catname{LieGrp}}
\newcommand{\HomSp}{\catname{HomSp}}
\newcommand{\HomMan}{\catname{HomMan}}
%\newcommand{\SmoothMan}{\catname{SmoothMan}}
\newcommand{\Man}{\catname{Man}}
\newcommand{\VectBund}{\catname{VectBund}}
\newcommand{\Klein}{\catname{Klein}}

\newcommand{\notzero}{\emptyset}
%\newcommand{\notzero}{{\!\not{\,0}}}

\renewcommand{\bar}[1]{\overline{#1}}
\renewcommand{\hat}[1]{\widehat{#1}}
\renewcommand{\tilde}[1]{\widetilde{#1}}

\newcommand{\meet}{\wedge}
\newcommand{\join}{\vee}
\newcommand{\eval}{\operatorname{eval}}
\newcommand{\cov}{\operatorname{cov}}
\newcommand{\var}{\operatorname{var}}
\newcommand{\std}{\operatorname{std}}
\newcommand{\esssup}{\operatorname{ess\,sup}}

\newcommand{\lleq}{\ \underline{\ll}\ }

\renewcommand{\a}{\mathbf{a}}
\newcommand{\va}{\mathbf{\check a}}
\newcommand{\q}{\mathbf{q}}
\newcommand{\vvarphi}{\check \varphi}

\newcommand{\daggerdagger}{{\dagger\dagger}}

\newcommand{\symdiff}{{\,\triangle\,}}

%\newcommand{\tom}[1]{\textbf{{\color{ForestGreen}#1}}}

\newcommand{\tom}[1]{{\textbf{\color{ForestGreen}\#}{\color{ForestGreen}\{\emph{#1}\}}}}
\newcommand{\jake}[1]{{\textbf{\color{Purple}\#}{\color{Purple}\{\emph{#1}\}}}}

%\newcommand{\tom}[1]{\textbf{{\color{ForestGreen}\#}}\{#1\}}

\author{Tom LaGatta and Jake Miller}
\title{Ergodic Ideals}

\date{July 2013}


\begin{document}

\maketitle

\textbf{Very rough draft. Please do not distribute.}
\begin{defn}[Properties of Conditional Expectation]\label{def:1}
  Let $\E(f|\C) : X \to R$ be the conditional expectation as defined
  in the previous section. Define $1 : X \to R$ to be a function
  equal to $1$ almost surely, and let $\mathbf{E}=\E[\cdot|\C]$ be the
  mapping $f \mapsto \E(f|\C)$.
  \begin{deflist}{width-text}
  \item[(I)] $\mathbf{E}$ is linear in $L^1(X,\X,P)$.
  \item[(II)] $\E(1|\C) = 1$
  \item[(III)] For each $C \in \C$ we have\footnote{This is because
      $\int\limits_C\E(\one_C \E(f|\C)|\C)dP=\int\limits_C \one_C
      \E(f|\C)dP=\int\limits_X \E(\one_C f | \C)dP=\int\limits_X
      \one_C fdP=\int\limits_C fdP$ \label{fn:1}} $\E(\one_C
    E(f|\C)|\C) = \E(f|\C)$ (so $\mathbf{E}$ is \emph{idempotent})
  \item[(IV)] If $f \geq 0$, then $\E(f|\C) \geq 0$
  \item[(V)] $f_n \rightarrow f$ in $L^1(X,\X,P) \implies \E(f_n|\C)
    \rightarrow \E(f|\C)$ in $L^1(X,\X,P)$
  \end{deflist}
\end{defn}

Our primary goal is to define conditional expectation in an order
theoretical context.

\begin{framed}
  Since $L^1(V,\X,\P)$ is complete, we must have that for any bounded
  set in $L^1(V,\X,\P)$ the supremum exists in $L^1(V,\X,\P)$. Thus
  $L^1(V,\X,\P)$ is a Dedekind complete (\textbf{Def:\ref{def:13}})
  Riesz space (\textbf{Def:\ref{def:6}}). This suggests that
  $\mathbf{E}$ should be a mapping on Dedekind complete Riesz
  spaces. Moreover, by (I), (III), and (IV), this mapping should be a
  \index{positive linear projection}positive linear projection on a
  Riesz subspace (\textbf{Def:\ref{def:8}}). (II) says that
  $\mathbf{E}$ should preserve weak order units
  (\textbf{Def:\ref{def:10}}), since the smallest order ideal
  (\textbf{Def:\ref{def:9}}) containing $1$ in $L^1(V,\X,\P)$ is the
  entire space, and this order ideal is a band (\textbf{Def:\ref{def:9}})
  since $L^1$ is Dedekind complete. Finally, (V) says that
  $\mathbf{E}$ should be order continuous (\textbf{Def:\ref{def:14}}).
\end{framed}

\part{Order Theory}

Order theory equips a set $V$ with a structure which defines an
\emph{ordering} of elements in the set using a \emph{relation} $\leq$,
and we write for $a \leq b$ to mean $a$ is \emph{below} $b$ and $a
\geq b$ to mean $a$ is \emph{above} $b$, for $a,b \in V$ (the terms
\emph{above} and \emph{below} are informal and depend on the order
structure).

A particular structure of this kind is called a \emph{partial order},
which as implied by the name does not restrict elements from being
neither above nor below each other --- in this structure, it is
\emph{not} required that either $a \leq b$ or $b \geq a$, $a,b \in
V$. Note that an order in which the above holds can still be a partial
order.

\begin{framed}
  \begin{defn}[partially ordered set]\label{def:2}
    A \emph{\index{partially ordered set}partially ordered set}, or
    \emph{\index{poset}poset}, is a set $V$ equipped with the ordering
    $\leq$, called \emph{partial order}, that satisfies the following
    properties:

    \begin{deflist}{width-text}
    \item $v \in V \implies v \leq v$ (\emph{reflexivity})
    \item If $v \leq w$ and $w \leq u$ then $v \leq u$ where $v,w,u
      \in V$ (\emph{transitivity})
    \item If $v \leq w$ and $w \leq v$ then $v = w$ where $v,w \in V$
      (\emph{anti-symmetry})
    \end{deflist}
  \end{defn}
\end{framed}

Naturally, an ordering gives rise to the idea of \emph{upper} and
\emph{lower} bounds for subsets of an ordered set, which are elements
(not necessarily contained in the set itself) that are above (below)
every element in a particular subset. A subset may also have a
particular type of upper bound called a \emph{supremum}, or a
particular type of lower bound called the \emph{infimum} --- these are
the least upper bounds and the greatest lower bounds respectively. The
anti-symmetry property in a partially ordered set implies the
uniqueness of the supremum and the infimum for each subset
\emph{provided that they exist}. This is because any least upper bound
is necessarily below (w.r.t the order) any other upper bound, implying
that any two non-unique suprema must be above and below each other
(since even though they are \emph{least} upper bounds, they are still
upper bounds) --- the anti-symmetry property of partial orderings
requires that in this case, both elements are equal. A similar
argument holds for greatest lower bounds.

A useful situation to highlight is when we want to know the supremum
and infimum of only two elements --- define the notation $a \join b$
for their supremum $\sup(\Set{x,y})$, or \emph{join}, of $a,b \in V$,
and $a \meet b$ for their infimum $\sup(\Set{x,y})$, or
\emph{meet}. This leads to a particular type of poset called a
\emph{lattice}.


\begin{framed}
  \begin{defn}[lattice]\label{def:3}

    A \emph{\index{lattice}lattice} is a poset $V$ where for any $v,w
    \in V$ we have $v \join w \in V$ and $v \meet w \in V$.

  \end{defn}
\end{framed}

For $V$ a lattice and $a,b,c \in V$, if $a \leq c $ and $b \leq c $,
we have that $a \join b \leq c $. Thus
\begin{equation}
  \label{eq:1}
  (u \meet w) \join (v \meet w) \leq w \join (u \meet v)
\end{equation}
since for any $u,v,w \in V$ we have that $u \meet w \leq w \join (u
\meet v)$ and $v \meet w \leq w \join (u \meet v)$. If we have
equality in (\ref{eq:1}), we get a \emph{distributive lattice}.


\begin{framed}
  \begin{defn}[distributive lattice]\label{def:4} %
    A \emph{\index{distributive lattice}distributive lattice} is a
    lattice where $w \join (u \meet v) = (w \meet u) \join (w \meet
    v)$.
  \end{defn}
\end{framed}

These tools can be used to equip vector spaces with an additional
structure. In particular, requiring that this enhanced vector space be
a lattice provides us with a \emph{Riesz space}, giving the foundation
on which we can construct a conditional expectation operator.

\begin{framed}
  \begin{defn}[ordered vector space]\label{def:5}
    An \emph{\index{ordered vector space}ordered vector space} V is a
    vector space over $\R$ with the following properties.

    Let $u,v,w \in V$:
    \begin{deflist}{width-text}
    \item V is partially ordered.
    \item $v \leq w \implies v + u \leq w + u$.
    \item If $v \leq w$ then $\alpha v \leq \alpha w$, $\alpha \geq
      0$, $\alpha \in \R$. (\emph{\index{homogeneity}homogeneity})
    \end{deflist}
  \end{defn}

\begin{defn}[Riesz space]\label{def:6}
  A \emph{\index{Riesz space}Riesz space} is an ordered vector space
  that is also a distributive lattice.
\end{defn}
\end{framed}

Subsets of Riesz spaces that arise from the lattice structure must be
described as well. Since Riesz spaces inherit the vector space
structure as well, we can talk about \emph{Riesz subspaces} --- these
are just ordinary subspaces with the additional property of being
closed under meets and joins. We can further classify these subspaces
into \emph{order ideals} and \emph{bands}.

From now on, we will assume that $V$ is a Riesz space. First, we must
describe a way of decomposing $v \in V$ into two different parts,
which will give us the concept of \emph{absolute value}.

\begin{framed}
  \begin{defn}[Absolute value]\label{def:7}
    For $v \in V$ we define $v^+=v \join 0$, $v^-=-v \join 0$, and
    $|v| = v \join -v$.
  \end{defn}


The following two theorems can be easily verified:

\begin{thm}[Properties of Decomposition]
  \label{thr:1}
  For $v,w \in V$, we have
  \begin{deflist}{width-text}
  \item
  \item[(i)] $v^+$ and $v^-$ are members of $V^+$, ${(-v)}^+=v^-$,
    ${(-v)}^-=v^+$, and $|-v|=|v|$.
  \item[(ii)] $v = v^+ - v^-$, $v^+ \meet v^-=0$, and $|v| = v^++v^-$
    (so $|v| \in V^+$).
  \item[(iii)] $0 \leq v^+ \leq |v|$ and $0 \leq v^- \leq
    |v|$. Furthermore, $-v^- \leq v \leq v^+$ and $|v|=0$ if and only
    if $v=0$.
  \item[(iv)] $v \leq w$ if and only if $v^+ \leq w^+$ and $v^- \geq
    w^-$.
  \end{deflist}
\end{thm}
Since $v \join w + u = (v + u) \join (w + u)$, the following theorem
follows from
\begin{align*}
  v \join w &= (v-w) \join 0 + w \\
  &= {(v-w)}^+ + w
\end{align*}

Similarly for $v \meet w = -{(v-w)}^+ v$.

\begin{thm}[Equalities]\label{thr:2}
  For $v,w \in V$, we have
  \begin{enumerate}
  \item\label{item:1} $v \join w + w \meet v = v + w$
  \item\label{item:2} $v \join w - w \meet v = |v - w|$
  \end{enumerate}
\end{thm}

\end{framed}

The following definitions are probably the most important tools used
in examining Riesz spaces.
\begin{framed}
  \begin{defn}[Riesz subspace]\label{def:8}
    A \emph{\index{Riesz subspace}Riesz subspace} $W$ of $V$ is a
    linear subspace of $V$ such that if $u,v \in W$ then $u \join v$
    and $u \meet v$ are also in $W$. (we can require only that $u
    \join v \in W$, since this implies that $u \meet v \in W$).
  \end{defn}
  We call a set $W \subset V$ \emph{solid} if $w \in W$, $v \in V$ and
  $|v| \leq |w|$, then $v \in W$. The absolute value allows us to talk
  about solid linear subspaces, since otherwise any solid linear
  subspace would expand into the entire space.

  \begin{defn}[order ideals and bands]\label{def:9}
    An \emph{\index{order ideal}order ideal} $I$ is a solid linear
    subspace of $V$. If every subset of this order ideal has its
    supremum (when it exists) in the order ideal itself, we call it a
    \emph{\index{band}band}. An order ideal generated by a non-empty
    subset $U$ of $V$ is the intersection of all order ideals
    containing $U$ --- similarly for a band. An order ideal
    \emph{generated} by $v \in V$ is the smallest order ideal
    containing $v$, which is the set $I_{v} = \Set{w \in V: |w| \leq
      |\alpha v|}$, $\alpha \in \R$.
  \end{defn}

  \begin{rem}\label{rem:1}
    It turns out that an order ideal is also a Riesz subspace. From
    \textbf{Def:\ref{def:8}} it suffices to show that $v \join w \in
    I$ for any $v,w \in V$. This follows immediately from
    \textbf{Thm:\ref{thr:2}}, which gives $v \join w =
    \frac{1}{2}(v+w+|v-w|) \leq |v| + |w| + |v - w|$.
  \end{rem}


\end{framed}

So every band is an order ideal, but not every order ideal is a
band. We can also distinguish between order ideals (bands) generated
by single elements, or \emph{principal order ideals
  (bands)}. Furthermore, we can examine those elements which can
generate order ideals (bands) equal to the space itself (note that the
entire space is necessarily a band).

\begin{framed}
  \begin{defn}[order unit]\label{def:10}
    An \emph{\index{order unit}order unit} is an element $e \in V$, $f
    > 0$ such that the order ideal generated by $e$ is $V$. Similarly,
    $e$ is a \emph{\index{weak order unit}weak order unit} if the band
    generated by $e$ is $V$.
  \end{defn}
  \begin{defn}[directed sets]\label{def:11}
    A non-empty subset $W$ of $V$ is called \emph{\index{upward
        directed}upward directed} if for any two elements $v,u \in W$
    there exists a $w \in W$ such that $w \geq v \join u$.A family of
    elements $v_{\alpha}$, $\alpha \in F$ where $F$ is some non-empty
    indexing set, is also said to be \emph{upward directed} if for
    each $a,b \in F$ there exists $c \in F$ such that $v_c \geq v_a
    \join v_b$. If $v = \sup({v_\alpha})$ then we say that
    $v_{\alpha}$ is \emph{upward directed towards v} and we write
    $v_{\alpha}\uparrow_{\alpha} v$. Downward directed is defined
    similarly. We define a \emph{\index{net}net} in V to be a function from a
    directed set $A$ to $V$, usually denoted by $\Set{v_{\alpha}}$ for
    $\alpha \in A$ and $v_{\alpha} \in V$. 
    \begin{rem}\label{rem:2}
      Since the band generated by a weak order unit $e \in V$ is
      $I_{e} = \Set{w \in V: |w| \leq |\alpha e|}$, $\alpha \in \R$,
      it follows that $e$ is a weak order unit if and only if $V =
      \Set{w \in V : (|w| \meet |\alpha e|) \uparrow_{\alpha}
        |w|}$. This will be useful when we define band projections
      (\textbf{Def:\ref{def:16}}).
    \end{rem}
  \end{defn}
\end{framed}

\begin{exa}[Ideal]\label{exa:1}
  Let $V=C[0,1]$, the set of real continuous functions on $[0,1]$ with
  the usual pointwise partial ordering. Define $I = \Set{f \in V :
    f(0) = 0}$. This is an ideal, since it is downward closed and for
  any $f,g \in I$ we have $\alpha f(0) + \beta g(0) = 0$. However, it
  is not a band. Define $f_n(x) = \inf_{x \in [0,1]}(nx, 1(x))$, where
  $1$ is the constant function equal to $1$ almost everywhere. Since
  $f_n(0) = 0$ for any $n$, $f_n \in I$. But if we take the supremum
  over all $n$ of $\Set{f_n}$ (as a function), we get that
  $\sup_n({f_n}) = 1$, and since $1(x) = 1 \neq 0$ the supremum of
  $\Set{f_n}$ is not contained in $I$. Thus $I$ is not a band.
\end{exa}

\begin{exa}[Band]\label{exa:2}
  Let $V=C[0,1]$ as above. Define $B = \Set{f \in V : f(x) = 0, x \in
    [0, \frac{1}{2}]}$. Clearly this is downward closed with respect
  to the partial ordering, and is also a linear subspace. Thus $B$ is
  an ideal. If $B$ is a band, then for every set of functions in $V$
  that possess a supremum, that supremum necessarily vanishes on
  $[0,\frac{1}{2}]$ as well. Suppose this were not the case. Then
  there exists an increasing net of functions $\Set{u_{\alpha}} \in B$
  converging to $u$, where $u(x_{0}) \neq 0 $ for some $x_0 \in
  [0,\frac{1}{2}]$. By the continuity of $u$ there exists a
  neighborhood $U$ of $x_0$ such that $u(x) \neq 0$ for $x \in U$. If
  we define a function $\gamma$ that is $1$ on $[0,1]/([0,\frac{1}{2}]
  \cap U)$ and $0$ everywhere else, then $u(1 - \gamma)$ is a smaller
  upper bound than $u$ since every $u_{\alpha}$ is necessarily $0$ on
  the set of points where $\gamma$ is $0$ anyway. Thus $B$ is a band.

\end{exa}

\begin{framed}
  \begin{defn}[positive cone]\label{def:12}
    The \emph{\index{positive cone}positive cone} of $V$ is the subset
    $V^+ = \Set{v \in V : 0 \leq v}$.
  \end{defn}

\begin{defn}[Dedekind complete]\label{def:13}
  A partially ordered set $U$ is said to be \emph{\index{Dedekind
      complete}Dedekind complete} if every non-empty subset in $U$
  that is bounded above (below) has a supremum (infimum) in $U$. A
  Riesz space is Dedekind complete when every non-empty upward
  directed subset of $V^+$ that is bounded above has a supremum.
\end{defn}

\begin{defn}[order continuous]\label{def:14}
  A linear operator $T : V \to W$ is said to be \emph{\index{order
      continuous}order continuous} if whenever $v_{\alpha} \downarrow
  0$ we have $T(v_{\alpha}) \downarrow 0$. This is equivalent to
  $v_{\alpha} \uparrow u$ implies $T(v_{\alpha}) \uparrow Tu$.
\end{defn}
\end{framed}

A nice example of a Dedekind complete space is the space of
real measurable functions $L^0(X,\R;\P)$ (identified up to $P$---measure zero).

\part{The Space of Measurable Functions}

Let $(X, \X)$ be a measurable space and consider the space of real
valued measurable functions on $X$, $\Meas(X,\R)$. Fixing a measure
$\P$ on $(X, \X)$ we define the \emph{\index{ideal}ideal} to be the
set $N_\P = \Set{A \in \X | \P(A) = 0}$. Let $L^0(X,R;\P)$ be the set
of equivalence classes of measurable functions up to $\P$-measure 0.

\begin{rem}[Relation to Order Ideal]\label{rem:4}
  Note that this ideal is an order ideal where the order is subset
  inclusion.
\end{rem}

Define an equivalence relation $f \sim g$ on $\Meas(X, \R)$ if $f(x) =
g(x)$ except on a set of measure 0 --- i.e, $\Set{x \in X : f(x) \neq
  g(x)} \in N_\P$. Then we have $L^0(X,R;\P) = \Meas(X,\R) / \sim$.

A \emph{\index{preorder}preorder} is a relation $\preceq $ that is
reflexive and transitive. Define a preorder $\preceq $ on the set
$\Meas(X,\R)$ such that $f \preceq g$ if $f \leq g$ except on a set of
measure zero, where $\leq $ is the usual pointwise partial
ordering. This means that $f \preceq g$ when $\set{x \in X | f(x) >
  g(x)} \in N_\P$. We say that this the preorder induced by $\P$.

In a way analagous to the partially ordered version, define a
\emph{pre-lattice\index{pre-lattice}} to be a preordered set with well
defined meets and joins, where $(f \meet g)(x) = \inf(f(x),g(x))$ and
$(f \join g)(x) = \sup(f(x),g(x))$ for all $x \in X$. Also define a
\emph{\index{preordered vector space}preordered vector space} in the
same way, analagous to the ordered vector space definition.

We say that $V$ is a \emph{\index{Preordered Riesz Space}Preordered
  Riesz Space} if it is \emph{Preordered Vector Space} with a
\emph{pre-lattice} structure.

\begin{pro}[Preordered Riesz Space]\label{pro:1}
  $\Meas(X, \R)$ is a preordered Riesz space under the preorder
  induced by $\P$ as described above.
\end{pro}

\begin{proof}
  First we verify reflexivity and transitivity.

  Reflexivity follows from the total ordering of the real line. For
  any $f \in \Meas(X, \R)$ we have that $\Set{x \in X | f(x) \geq
    f(x)} = \emptyset \in N_P$. Thus $f \preceq f$.

  For transitivity, suppose $f \preceq g$ and $g \preceq h$ and define

\begin{align*}
  A &= \Set{x \in X | f(x) > g(x)} \\
  B &= \Set{x \in X | g(x) > h(x)}
\end{align*}

By definition, $A,B \in N_P$. Since $A \cap B \subset A$ we have that
$A \cap B \in N_P$. But this says that $f(x) > h(x)$ only on a set of
measure zero. Thus $\preceq $ is transitive.

That it is a preordered vector space follows similarly.

\end{proof}

\begin{pro}[Distributive]\label{pro:2}
  A preordered Riesz space is a distributive pre-lattice.
\end{pro}

\begin{proof}
  This follows directly from $\R$ being a distributive lattice. For
  all $x \in X$ we have
  \begin{align*}
    (h \meet (f \join g))(x) &= \inf(h(x), \sup(f(x),g(x))) \\
    &= \sup(\inf(h(x),f(x)),\inf(h(x),g(x))) \\
    &= ((h \meet f) \join (h \meet g))(x)
  \end{align*}
  and since each supremum and infimum is well-defined in $\R$ we have
  our result.
\end{proof}

\begin{pro}[Quotient Space]\label{pro:3}
  Define $f \sim g$ if $f \preceq g$ and $g \succeq f$. Then the
  quotient space $\Meas(X,\R)/\sim$ is a Riesz space.
\end{pro}

\begin{proof}
  $\Meas(X, \R)/\sim$ consists of equivalence classes of the form $[f]
  = \Set{g \in \Meas(X, \R) | f \sim g }$. Define an ordering on
  equivalence classes as follows: $[f] \leq_P [g]$ if $f(x) \leq g(x)$
  except on a set of measure $0$. From the preorder on $\Meas(X,\R)$
  we know that this ordering is reflexive and transitive.  If $[f]
  \leq_P [g]$ and $[g] \leq_P [f]$ we must have that $f(x) = g(x)$
  except on a set of measure $0$. But this means that $f \sim g$, and
  thus $[f] = [g]$. So $\leq_P$ is anti-symmetric and thus is a
  partial ordering on the quotient space. The preorder also gives that
  $\Meas(X,\R)/\sim$ is an ordered vector space and a lattice under
  the ordering $\leq_P$. It follows that $\Meas(X,\R)/\sim$ is a Riesz
  space.
\end{proof}

Define $L^{\infty}=L^{\infty}(X,\R)$ to be the space of essentially
bounded functions, which is a subset of $L^0$.

\begin{pro}[$L^{\infty}$ is an Order Ideal]\label{pro:4}
  $L^{\infty}$ is an order ideal in $\Meas(X,\R)$. In particular, it
  is the principal ideal generated by any constant function $\rho \in
  \Meas(X,\R)$.
\end{pro}

\begin{proof}
  Let $I_{\rho}$ be the ideal generated by $\rho$. This is the
  smallest ideal containing $\rho$, given explicitly by $\Set{g \in
    \Meas(X, \R) : |g| \leq |\alpha \rho|}$ for some $\alpha \in
  \R$. But $g \in I_{\rho}$ if and only if the set $A = \Set{x \in X :
    |g|(x) > |\alpha \rho|(x)} = \Set{x \in X : |g|(x) > |C|} \in N_P$
  for some constant $C$. Thus $I_{\rho} = L^{\infty}$.
\end{proof}

\begin{rem}[$L^{\infty}$ not a Band]\label{rem:5}
  $L^{\infty}$ is not a band in $L^0$. To see this, we only
  need to find a set of essentially bounded functions with a supremum
  that is not essentially bounded. Since the set of simple functions
  is a subspace of $L^{\infty}$, we only need to find a measurable
  function that is in $L^p$ for some $p$ but not in $L^{\infty}$,
  since functions in $L^P$ can be approximated by simple
  functions. $\frac{1}{\sqrt{x}}$ on $(0,1)$ is an example, since it
  is integrable on $(0,1)$ but blows up close to $0$. 
\end{rem}

\begin{pro}[$L^{0}$ is Dedekind Complete]\label{pro:5}
$L^{0}$ is a Dedekind complete space when $X$ is finite or countable
union of sets with finite measure --- every bounded subset of the
positive cone $(L^0)^+$ has a supremum. 
\end{pro}

\begin{proof}
\end{proof}

\begin{pro}[$L^{p}$ is an Ideal]
\end{pro}

\begin{proof}
\end{proof}

\begin{pro}[Ideals in $L^0$ are Dedekind complete]
\end{pro}


\part{Order Conditional Expectation}

\begin{defn}[Properties of Conditional Expectation]\label{def:1}
  Let $\E(f|\C) : X \to R$ be the conditional expectation as defined
  in the previous section. Define $1 : X \to R$ to be a function
  equal to $1$ almost surely, and let $\mathbf{E}=\E[\cdot|\C]$ be the
  mapping $f \mapsto \E(f|\C)$.
  \begin{deflist}{width-text}
  \item[(I)] $\mathbf{E}$ is linear in $L^1(X,\X,P)$.
  \item[(II)] $\E(1|\C) = 1$
  \item[(III)] For each $C \in \C$ we have\footnote{This is because
      $\int\limits_C\E(\one_C \E(f|\C)|\C)dP=\int\limits_C \one_C
      \E(f|\C)dP=\int\limits_X \E(\one_C f | \C)dP=\int\limits_X
      \one_C fdP=\int\limits_C fdP$ \label{fn:1}} $\E(\one_C
    E(f|\C)|\C) = \E(f|\C)$ (so $\mathbf{E}$ is \emph{idempotent})
  \item[(IV)] If $f \geq 0$, then $\E(f|\C) \geq 0$
  \item[(V)] $f_n \rightarrow f$ in $L^1(X,\X,P) \implies \E(f_n|\C)
    \rightarrow \E(f|\C)$ in $L^1(X,\X,P)$
  \end{deflist}
\end{defn}

\begin{framed}
  Since $L^1(V,\X,\P)$ is complete, we must have that for any bounded
  set in $L^1(V,\X,\P)$ the supremum exists in $L^1(V,\X,\P)$. Thus
  $L^1(V,\X,\P)$ is a Dedekind complete (\textbf{Def:\ref{def:13}})
  Riesz space (\textbf{Def:\ref{def:6}}). This suggests that
  $\mathbf{E}$ should be a mapping on Dedekind complete Riesz
  spaces. Moreover, by (I), (III), and (IV), this mapping should be a
  \index{positive linear projection}positive linear projection on a
  Riesz subspace (\textbf{Def:\ref{def:8}}). (II) says that
  $\mathbf{E}$ should preserve weak order units
  (\textbf{Def:\ref{def:10}}), since the smallest order ideal
  (\textbf{Def:\ref{def:9}}) containing $1$ in $L^1(V,\X,\P)$ is the
  entire space, and this order ideal is a band (\textbf{Def:\ref{def:9}})
  since $L^1$ is Dedekind complete. Finally, (V) says that
  $\mathbf{E}$ should be order continuous (\textbf{Def:\ref{def:14}}).
\end{framed}

Our primary goal is to define conditional expectation in an order
theoretical context.

\begin{defn}[Conditional Expectation on Riesz Spaces]\label{def:15}

  The \emph{\index{order conditional expectation}order conditional
    expectation} $\E_W: V \to W$, where $V$ is a Dedekind
  complete Riesz space with a weak order unit and $W \subset V$ is a
  Dedekind complete Riesz subspace, satisfies the following
  properties:

  \begin{deflist}{width-text}
  \item[(I)] $\E_W$ is a \emph{positive linear projection}.
  \item[(II)] $\E_W$ is \emph{order continuous}.
  \item[(III)] $\E_W$ preserves \emph{weak order units}.m
  \end{deflist}
\end{defn}

\begin{framed}
  It follows that there exists a weak order unit in $e \in V$ such
  that $e = Te$ if and only if $Td$ is a weak order unit for every
  weak order unit $d \in V$. So a positive, order continuous, linear
  projection from a Dedekind complete Riesz space to a subspace of a
  Dedekind complete Riesz space is a conditional expectation if it
  preserves weak order units.
\end{framed}

In order to formulate the Maximal Ergodic Theorem in terms of Riesz
spaces, we need one more definition.

\begin{defn}[Band Projection]\label{def:16}
  The map $J_w(u) = \sup_n (u^+ \meet nw) - \sup_n(u^- \meet nw)$ for
  all $w \in V$ is called the \emph{\index{band projection}band
    projection} onto the band generated by $w$, ie the band $B_w =
  \Set{v \in V | (|v| \meet nw) \uparrow_n |v|}$ (see
  \textbf{Rem:\ref{rem:2}}) generated by $w$ when $w$ is in the
  positive cone $V^+$.
\end{defn}

Restricting $\E_W$ to a band projection onto the band generated
by a \emph{\index{maximal operator}maximal operator} $M$ will give us
the Maximal Ergodic Theorem.

\begin{thm}[Maximal Ergodic Theorem on Riesz Spaces]\label{thr:3}

  Let $V$ be a Riesz space with a weak order unit $e$. Let
  $\E_W$ be the conditional expectation operator on V such that
  $e = \E_W e$, and define the operator $T : V \to V$ to be
  positive with $\E_W|Tf| \leq \E_W|f|$, $f \in
  \E_W$. Define the \emph{\index{nth maximal operator}nth
    maximal operator}, for $n \geq 1$
  \begin{equation}\label{eq:2}
    M_n(f) = \sum\limits_{k=0}^{0} T^k f \join \sum\limits_{k=0}^{1} T^k f \join \cdots \join \sum\limits_{k=0}^{n-1} T^k f = \bigvee_{i=0}^n \sum\limits_{k=0}^{i-1} T^k f
  \end{equation}

  Now define the \emph{\index{maximal band}maximal band} $M(f)$ to be
  the band generated by $\Set{M_n{(f)}^+ | n \in \NN}$, and $J_{M(f)}$
  to be the band projection onto the maximal band. Then we have

\begin{equation}\label{eq:3}
  \E_W(J_{M(f)}f) \geq 0
\end{equation}
\end{thm}

\begin{framed}
  First we prove that that for each nth maximal operator we have
  something similar to \eqref{eq:3}. Each nth maximal operator is
  positive and thus we can write it in terms of a band projection
  $J_{M_n{(f)}^+}$ on the band generated by the nth maximal operator
  itself. We then note that the sequence of band projections
  associated with each maximal operator is upward directed to the band
  projection onto the maximal band. Once we have that
  $\E_W(J_{M_n{(f)}^+}f) \geq 0$, we can then use the order
  continuity of $\E_W$ to complete the proof.
\end{framed}

\begin{proof}\label{prf:max-erg}

  Since $T$ is positive and
  \begin{equation}
    \label{eq:4}
    M_n(f) \geq \sum\limits_{k=0}^{j-1} T^k f
  \end{equation}

  for $j = 0,\dots,n$, we must have that for the same $j$ values,

  \begin{equation}
    \label{eq:5}
    TM_n(f) \geq T\sum\limits_{k=0}^{j-1} T^k f = \sum\limits_{k=0}^{j-1} T^{k+1}f
  \end{equation}

  Since $TM_n(f) = T\bigvee_{i=0}^n\sum\limits_{k=0}^{i-1} T^k f$ we
  have that

  \begin{equation}
    \label{eq:6}
    f + TM_n(f) = f + T\bigvee_{i=0}^n\sum\limits_{k=0}^{i-1} T^k f 
  \end{equation}

  and then \eqref{eq:5} implies that

  \begin{align}
    \label{eq:7}
    f + TM_n(f) &\geq f + \bigvee_{i=0}^n\sum\limits_{k=0}^{i-1} T^{k+1}f  \\
    \label{eq:8}
    &= \bigvee_{i=0}^n( f + \sum\limits_{k=0}^{i-1} T^{k+1}f)  \\
    \label{eq:9}
    &= \bigvee_{i=1}^{n+1}( \sum\limits_{k=0}^{i-1} T^{k}f)
  \end{align}

  By \eqref{eq:4} we know that \eqref{eq:9} must be larger than
  $\sum\limits_{k=0}^{j-1} T^{k}f$ for $j = 1,\dots,n+1$. This means
  we can write

  \begin{align}
    \label{eq:10}
    f + TM_n(f) &\geq \bigvee_{i=1}^{n}\sum\limits_{k=0}^{i-1} T^{k}f  \\
    \label{eq:11}
    f &\geq \bigvee_{i=1}^{n}\sum\limits_{k=0}^{i-1} T^{k}f - TM_n(f)
  \end{align}


  Also, we have that the nth maximal operator is always positive,
  since the case of $n=0$ forces $0$ to be the smallest possible upper
  bound. This means that we can express $M_n(f)$ as a band projection
  in two different form, the trivial one

  \begin{equation}
    \label{eq:12}
    M_n(f) = J_{M_n{(f)}^+}M_n(f) 
  \end{equation}

  and another

  \begin{align}
    \label{eq:13}
    M_n(f) &= 0 \join \bigvee_{i=1}^n\sum\limits_{k=0}^{i-1} T^{k}f \\
    \label{eq:14}
    &= J_{M_n{(f)}^+}\bigvee_{i=1}^n\sum\limits_{k=0}^{i-1} T^{k}f
  \end{align}

  Thus we can apply $J_{M_n{(f)}^+}$ to \eqref{eq:11} and substitute,
  yielding

  \begin{align}
    \label{eq:15}
    J_{M_n{(f)}^+}(f) &\geq J_{M_n{(f)}^+}(\bigvee_{i=1}^{n}\sum\limits_{k=0}^{i-1} T^{k}f - TM_n(f)) \\
    \label{eq:16}
    &= M_n(f)-(J_{M_n{(f)}^+}TM_n(f))
  \end{align}

  Finally, applying $\E_W$, using that $\E_W J=J\E_W$
  and $\E_W T(f) \leq \E_W(f) $ (recall that $T$ is a
  positive operator) gives
  \begin{align*}
    \label{eq:17}
    \E_W J_{M_n{(f)}^+}(f)&\geq \E_W M_n(f)-\E_W(J_{M_n{(f)}^+}TM_n(f)) \\
    &\geq \E_W M_n(f)-\E_W(J_{M_n{(f)}^+}M_n(f)) \\
    &= \E_W(M_n(f)-(J_{M_n{(f)}^+}M_n(f))) \\
    &= \E_W(0) \\
    &= 0
  \end{align*}

\end{proof}

\printindex
\end{document}
		
		
		
		
		
		
		
		
