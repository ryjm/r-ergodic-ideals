\documentclass[letterpaper,10pt,oneside,onecolumn,reqno]{amsart}
%leqno = left hand equation numbering
%oneside vs. twoside = how many pages are you looking at at a time?
%onecolumn vs. twocolumn = obvious

%%% Packages
\usepackage{amsmath, amssymb}
\usepackage[left=1in,top=1in,right=1in,bottom=1in]{geometry}
\usepackage{setspace}
\usepackage{hyperref}
\usepackage{enumerate}
\usepackage{wrapfig}
\usepackage{graphicx}
\usepackage[all]{xy}
\usepackage{color}
\usepackage[usenames,dvipsnames]{xcolor}
\usepackage{refcheck}

\onehalfspace
\setcounter{tocdepth}{2}	% If this equals 1, the table of contents does not include subsections
%\includeonly{}

%\def\singlespaced{\baselineskip=\normalbaselineskip}


%%% Sets
\newcommand{\A}{\mathcal A}
\newcommand{\B}{\mathcal B}
\newcommand{\C}{\mathcal C}
\newcommand{\D}{D}
\newcommand{\E}{\mathbb E}
\newcommand{\F}{\mathcal F}
\renewcommand{\H}{\mathcal H}
\newcommand{\I}{\mathcal I}
\newcommand{\K}{\mathcal K}
\newcommand{\M}{\mathcal M}
%\newcommand{\MM}{\operatorname{M}}
\newcommand{\N}{\mathcal N}
\newcommand{\NN}{\mathbb N}
\renewcommand{\P}{\mathbb P}
\newcommand{\PP}{\mathbb P}
\newcommand{\Q}{\mathbf Q}
\newcommand{\R}{\mathbb R}
\newcommand{\T}{\mathcal T}
\newcommand{\V}{\mathcal V}
\newcommand{\W}{\mathcal W}
\newcommand{\X}{\mathcal X}
\newcommand{\Z}{\mathbb Z}

\renewcommand{\c}{{\operatorname{c}}}
\newcommand{\e}{\mathrm{e}}				% Euler's constant.
\renewcommand{\d}{\mathrm{d}}				% Differential operator sans space.
\newcommand{\sd}{\, \mathrm{d}}		% Differential operator with space.
\renewcommand{\i}{\mathrm{i}}				% Square root of $-1$.
\newcommand{\oo}{\infty}
\newcommand{\Var}{\operatorname{Var}}

\newcommand{\tr}{\operatorname{tr}}

\newcommand{\ent}{\operatorname{ent}}

\newcommand{\arginf}{\operatorname{arginf}}
\newcommand{\argsup}{\operatorname{argsup}}

% Environments
\theoremstyle{definition}
\newtheorem{thm}{Theorem}
\newtheorem{cor}[thm]{Corollary}
\newtheorem{lem}[thm]{Lemma}
\newtheorem{pro}[thm]{Proposition}
\newtheorem{conj}[thm]{Conjecture}
\newtheorem{defn}[thm]{Definition}
\newtheorem{ass}[thm]{Assumption}
\newtheorem{rem}[thm]{Remark}
\newtheorem{exa}[thm]{Example}
%\theoremstyle{alpha}
\newtheorem{hyp}{Hypothesis}


\newcommand{\End}{\operatorname{End}}
\newcommand{\Hom}{\operatorname{Hom}}
\newcommand{\Aut}{\operatorname{Aut}}
\newcommand{\InnAut}{\operatorname{InnAut}}

\newcommand{\Euc}{\operatorname{Euc}}
\renewcommand{\O}{\operatorname{O}}
\newcommand{\SO}{\operatorname{SO}}

\newcommand{\catname}[1]{{\normalfont\textbf{#1}}}
\newcommand{\DynSys}{\catname{DynSys}}
\newcommand{\Set}{\catname{Set}}
\newcommand{\Meas}{\catname{Meas}}
\newcommand{\Top}{\catname{Top}}
\newcommand{\LieGrp}{\catname{LieGrp}}
\newcommand{\HomSp}{\catname{HomSp}}
\newcommand{\HomMan}{\catname{HomMan}}
%\newcommand{\SmoothMan}{\catname{SmoothMan}}
\newcommand{\Man}{\catname{Man}}
\newcommand{\VectBund}{\catname{VectBund}}
\newcommand{\Klein}{\catname{Klein}}

\newcommand{\notzero}{\emptyset}
%\newcommand{\notzero}{{\!\not{\,0}}}

\renewcommand{\bar}[1]{\overline{#1}}
\renewcommand{\hat}[1]{\widehat{#1}}
\renewcommand{\tilde}[1]{\widetilde{#1}}

\newcommand{\meet}{\wedge}
\newcommand{\join}{\vee}
\newcommand{\eval}{\operatorname{eval}}
\newcommand{\cov}{\operatorname{cov}}
\newcommand{\var}{\operatorname{var}}
\newcommand{\std}{\operatorname{std}}
\newcommand{\esssup}{\operatorname{ess\,sup}}

\newcommand{\lleq}{\ \underline{\ll}\ }

\renewcommand{\a}{\mathbf{a}}
\newcommand{\va}{\mathbf{\check a}}
\newcommand{\q}{\mathbf{q}}
\newcommand{\vvarphi}{\check \varphi}

\newcommand{\daggerdagger}{{\dagger\dagger}}

\newcommand{\symdiff}{{\,\triangle\,}}

%\newcommand{\tom}[1]{\textbf{{\color{ForestGreen}#1}}}

\newcommand{\tom}[1]{{\textbf{\color{ForestGreen}\#}{\color{ForestGreen}\{\emph{#1}\}}}}
\newcommand{\jake}[1]{{\textbf{\color{ForestGreen}\#}{\color{ForestGreen}\{\emph{#1}\}}}}

%\newcommand{\tom}[1]{\textbf{{\color{ForestGreen}\#}}\{#1\}}

\author{Tom LaGatta and Jake Miller}
\title{Ergodic Ideals}
\date{February 2013}



\begin{document}

\maketitle

	\textbf{Very rough draft. Please do not distribute.}

	Goal: ergodic averages from the structural viewpoint. 

	Measurable space $(X, \X)$, along with a measurable function $f : X \to X$. Assume that $\N$ is an ideal on $X$,\footnote{\tom{what is an ideal?}} and that the action of $f$ is ergodic with respect to $\N$.\footnote{\tom{what does this mean?}}
	
	
%	$\frac 1 N \sum_{n=1}^N \int_X \f( a^n(x) ) \, \P(\d x)$.
	
%	$\frac 1 N \sum_{n=1}^N f(a^n(x))$

	\section{Measure Formulation}

	Let $(X,\X)$ be a measurable space. \tom{Describe measurable subsets as ``events''.}
	
	Let $T = \NN$ be the natural numbers, and let $a^t : X \to X$ be a measurable action on the space.\footnote{That is, $a^{t'} \circ a^t = a^{t' + t}$ and $a^{0} = 1_X$, the identity map on $X$.} We define the inverse maps by $a^{-t} : \X \to \X$, noting that they go in the opposite direction. Formally, $a : T \to \End(X)$.
	
	
	
	Let $\P$ be a stationary, ergodic measure on $X$. As a measure, $\P : \X \to [0,1]$ is a countably additive function.
	
	Stationarity means that $a^t_* \P = \P$.\footnote{That is, $\P \circ a^{-t} = \P$.} Ergodic means that
		$$\mbox{if $\P( A \,\triangle\, a^{-t} A ) = 0$ for all $t \in T$, then $\P(A) = 0$ or $\P(A^c) = 0$.}$$

	\tom{What is a conditional expectation? $\E(f|\C)$ is a function $X \to \R$ which satisfies certain properties.} 
	
	
	

	\tom{Get to Birkhoff-Khinchin ergodic theorem}
	
		\begin{thm}[Birkhoff-Khinchin Ergodic Theorem]
			
		\end{thm}
		\begin{proof}
			
		\end{proof}
		
		\begin{cor}[Strong Law of Large Numbers]
		
		\end{cor}

	\section{Ideal Formulation}

	A subset $\N \subseteq \X$ is called an ideal if it is closed under countable unions, and is downward closed. That is, if $\{ N_i \} \subseteq \N$ is a countable set of measurable events, then $\bigcup N_i \in \N$; and if $N \in \N$ and $A \in \X$ with $A \subseteq N$, then $A \in \N$.

	We say that $\N$ is an ergodic ideal with respect to the $T$-action described by $a^t$ if:
		$$\mbox{if $A \,\triangle\, a^{-t} A \in \N$ for all $t \in T$, then $A \in \N$ or $A^c \in \N$.}$$

\section{Order Formulation}
We would like to construct an analogue of conditional expectation in the context of Riesz spaces. First we list some properties of the usual conditional expectation.
\begin{defn}[Properties of Conditional Expectation]
Let $\E(f|\C) : X \to R$ be the conditional expectation as defined in the previous section. Define $\bar{1} : X \to R$ to be a function equal to $1$ almost surely, and let $\mathbf{E}=\E[\cdot|\C]$ be the mapping $f \mapsto \E(f|\C)$.
\begin{deflist}{width-text}
\item[(I)] $\mathbf{E}$ is linear in $L^1(X,\X,P)$. 
\item[(II)] $\E(\bar{1}|\C) = \bar{1}$
\item[(III)] For each $C \in \C$ we have\footnote{This is because $\int\limits_C\E(\one_C \E(f|\C)|\C)dP=\int\limits_C \one_C \E(f|\C)dP=\int\limits_X \E(\one_C f | \C)dP=\int\limits_X \one_C fdP=\int\limits_C fdP$ } $\E(\one_C E(f|\C)|\C) = \E(f|\C)$ (so $\mathbf{E}$ is \emph{idempotent}) 
\item[(IV)] If $f \geq 0$, then $\E(f|\C) \geq 0$
\item[(V)] $f_n \rightarrow f$ in $L^1(X,\X,P) \implies \E(f_n|\C) \rightarrow \E(f|\C)$ in $L^1(X,\X,P)$
\end{deflist}
\end{defn}

With these properties in mind, we now define some order theoretical terms.

	
\begin{defn}
A \emph{partially ordered set}, or \emph{poset}, is a set $X$ equipped with the ordering $\leq$ that satisfies the following properties:

\begin{deflist}{width-text}
\item $x \in X \implies x \leq x$ (\emph{reflexivity})
\item If $x \leq y$ and $y \leq z$ then $x \leq z$, $x,y,z \in X$ (\emph{transitivity})
\item If $x \leq y$ and $y \leq x$ then $x = y$, $x,y \in X$ (\emph{anti-symmetry})
\end{deflist}
\end{defn}

\begin{defn}
An \emph{ordered vector space} X is a vector space with the following properties:

\begin{deflist}{width-text}
\item X is partially ordered.
\item Vector space addition preserves order.
\item If $x \leq y$ then $\alpha x \leq \alpha y$. (\emph{homogeneity})
\end{deflist}
\end{defn}

\begin{defn}

A \emph{lattice} is a poset $X$ where every two element subset has both a supremum and an infimum. For $x, y \in X$ denote $\sup(x,y)$ and $\inf(x,y)$ by $x \vee y$ and $x \wedge y$ respectively. 
\end{defn}

\begin{defn}
For $f \in X$, where X is a lattice, we define $f^+=f \wedge 0$, $f^-=-f \wedge 0$, and $|f| = f \vee -f$
\end{defn}

\begin{defn}
A \emph{distributive lattice} is a lattice where $x \wedge (y \vee w) = (x \wedge y) \vee (x \wedge w)$. It can be shown that we can exchange $\wedge$ and $\vee$ as needed.
\end{defn}

\begin{defn}
  A \emph{Riesz space} is an ordered vector space that is also a lattice. In particular, this lattice is necessarily a distributive lattice.

$L^1(X,\X,\P)$ is a Riesz space under the pointwise partial order ($f \leq g$ when $f(x) \leq g(x)$).

\end{defn}
In the following, let $E$ be a Riesz space.

\begin{defn}
A \emph{Riesz subspace} $S$ of $E$ is a (in the sense of vector spaces) linear subspace of $E$ such that if $f,g \in S$ then $f \vee g$ and $f \wedge g$ are also in $S$.
\end{defn}

\begin{defn}
 A \emph{solid subset} is a set $A \subset E$ such that if $f \in A$ and $|g| \leq |f|$, then $g \in A$.
\end{defn}

\begin{defn}
An \emph{ideal} is a solid linear subspace of $E$. An ideal \emph{generated} by $f\in E$ is the smallest ideal containing $f$, which is the set $I_{f} = \Set{g \in E: |g| \leq |\alpha f|}$, $\alpha \in \R$. 
\end{defn}

\begin{defn}
A \emph{band} is an ideal of $E$ such that every subset of the ideal has its supremum (when it exists in $E$) in the ideal. A band \emph{generated} by $f\in E$ is the smallest band containing $f$ (and thus contains $I_{\alpha}$ as well), and is called a \emph{principal band}. An ideal generated by a non-empty subset $D$ of $E$ is the intersection of all ideals containing $D$ - similarly for a band.
\end{defn}

\begin{defn}
An \emph{order unit} is an element $f \in E$, $f > 0$ such that the ideal generated by $f$ is $E$. Similarly, $f$ is a \emph{weak order unit} if the band generated by $f$ is $E$.
\end{defn}


\begin{defn}
A set $H$ is called \emph{upward directed} if for any two elements $x,y \in H$ there exists a $z \in H$ such that $z \geq x \vee y$. A set $H$ is called \emph{downward directed} if for any two elements $x,y \in H$ there exists a $z \in H$ such that $z \leq x \wedge y$. A family of elements $f_{\alpha}$, $\alpha \in F$ where $F$ is some non-empty set, is also said to be \emph{upward directed} if for each $a,b \in F$ there exists $c \in F$ such that $f_c \geq f_a \vee f_c$. If $f = \sup({f_\alpha})$ then we say that $f_{\alpha}$ is \emph{upward directed towards f} and we write $f_{\alpha}\uparrow f$. Downward directed is defined similarly. 
\end{defn}

\begin{defn}
A linear operator $T : E \to D$ is said to be \emph{order continuous} if whenever $f_{\alpha} \downarrow 0$ we have $T(f_{\alpha}) \downarrow 0$.
\end{defn}
\begin{defn}
A linear operator $T : E \to D$ is said to be \emph{positive} if whenever $f \geq 0$ we have $T(f) \geq 0$.
\end{defn}

\begin{defn}
A \emph{positive cone} of $E$ is a subset $E^+ = \Set{f \in E : 0 \leq f}$.
\end{defn}


\begin{defn}[Dedekind complete]
A partially ordered set $F$ is said to be \emph{Dedekind complete} if every non-empty subset in $F$ that is bounded above (below) has a supremum (infimum) in $F$. A Riesz space is Dedekind complete when every non-empty upward directed subset of $E^+$ that is bounded above has a supremum.
\end{defn}


Note that since $L^1(X,\X,\P)$ is complete, we must have that for any bounded set in $L^1(X,\X,\P)$ the supremum exists in $L^1(X,\X,\P)$. Thus $L^1(X,\X,\P)$ is a Dedekind complete Riesz space. This suggests that $\mathbf{E}$ should be a mapping on Dedekind complete Riesz spaces. Moreover, by (I), (III), and (IV), this mapping should be a positive linear projection on a Riesz subspace. (II) says that $\mathbf{E}$ should preserve weak order units, since the smallest ideal containing $\bar{1}$ in $L^1(X,\X,\P)$ is the entire space, and this ideal is a band since $L^1$ is Dedekind complete. Finally, (V) says that $\mathbf{E}$ should be order continuous. 

\begin{defn}[Conditional Expectation on Riesz Spaces]
The \emph{conditional expectation} $\mathbf{E}: A \to B$, where $A$ is a Dedekind complete Riesz space with a weak order unit and $B \subset A$ is a Dedekind complete Riesz subspace, satisfies the following properties:

\begin{deflist}{width-text}
\item[(I)] $\mathbf{E}$ is a \emph{positive linear projection}.
\item[(II)] $\mathbf{E}$ is \emph{order continuous}.
\item[(III)] $\mathbf{E}$ preserves \emph{weak order units}.
\end{deflist}

It follows that there exists a weak order unit in $e \in A$ such that $e = Te$ if and only if $Td$ is a weak order unit for every weak order unit $d \in A$. So a positive, order continuous, linear projection from a Dedekind complete Riesz space to a subspace of a Dedekind complete Riesz space is a conditional expectation if it preserves weak order units.
\end{defn}

In order to formulate the Maximal Ergodic Theorem in terms of Riesz spaces, we need one more definition. 

\begin{defn}[Band Projection]
The map $J_b(f) = \sup_n (f^+ \wedge nb) - \sup_n(f^- \wedge nb)$ for all $f \in E$ is called the \emph{band projection} onto the band generated by $b$, ie the band $B_b = \Set{f \in E | (|f| \wedge nb) \uparrow |f|}$ generated by $b$ when $b$ is in the positive cone $E^+$. 
\end{defn}

Restricting $\mathbf{E}$ to a band projection onto the band generated by a \emph{maximal operator} $M$ will give us the Maximal Ergodic Theorem.


\begin{thm}[Maximal Ergodic Theorem on Riesz Spaces]
Let $E$ be a Riesz space with a weak order unit $e$. Let $\mathbf{E}$ be the conditional expectation operator on E such that $e = \mathbf{E}e$, and define the operator $T : E \to E$ to be positive with $\mathbf{E}|Tf| \leq \mathbf{E}|f|$, $f \in \mathbf{E}$. Define the \emph{nth maximal operator}, for $n \geq 1$
\begin{equation}
M_n(f) = \sum\limits_{k=0}^{0} T^kf \vee \sum\limits_{k=0}^{1} T^kf \vee \cdots \vee \sum\limits_{k=0}^{n-1} T^kf
\end{equation}

Now define the \emph{maximal band} $M(f)$ to be the band generated by $\Set{M_n(f)^+ | n \in \NN}$, and $J_{M(f)}$ to be the band projection onto the maximal band. Then we have

\begin{equation}
\mathbf{E}(J_{M(f)}f) \geq 0
\end{equation}
\end{thm}

\end{document}		
		
		
		
		
		
		
		
		
		
		
		
		
